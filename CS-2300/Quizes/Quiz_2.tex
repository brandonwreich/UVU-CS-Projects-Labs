\documentclass{article}
\usepackage{verbatim}

\title{Quiz 2}
\author{Brandon Reich}
\date{23 January 2026}

\begin{document}

\maketitle

\subsection*{1.} 
The code block below verifies $\forall x \forall y\ check\_it(x,y)$ because it makes sure that every combination within the domain resolve to true.

\begin{verbatim}
result = True

for x in range(1, 1001):
  for y in range(1, 1001):
    if not check_it(x,y):
      result = False
      break

  if not result:
    break

print(result)
\end{verbatim}

\subsection*{2.}
The code block below verifies $\forall x \exists y\ check\_it(x,y)$ because for each x it searches for one y that works. If found, it moves on to the next x. If there is no working y then it exits and returns false. If every x has a working y then it returns true.

\begin{verbatim}
result = True

for x in range(1, 1001):
  for y in range(1, 1001):
    if check_it(x,y):
      result = True
      break
    else:
      result = False

  if not result:
    break

print(result)

\end{verbatim}

\subsection*{3.}
The code block below verifies $\exists x \forall y\ check\_it(x,y)$ because it searches for an x where all y values return True. If a y value is found false it moves on to the next x value. If all y values pass it exits and returns true. Otherwise it will return false.

\begin{verbatim}
result = False

for x in range(1, 1001):
  for y in range(1, 1001):
    if not check_it(x,y):
      result = False
      break
    else:
      result = True

  if result:
    break

print(result)
\end{verbatim}

\subsection*{4.}
The code block below verifies $\exists x \exists y\ check\_it(x,y)$ because as soon as it finds an x and y value that returns true, the loops exit and return true. Otherwise it returns false.

\begin{verbatim}
result = False

for x in range(1, 1001):
  for y in range(1,1001):
    if check_it(x,y):
      result = True
      break

  if result:
    break

print(result)
\end{verbatim}

\end{document}